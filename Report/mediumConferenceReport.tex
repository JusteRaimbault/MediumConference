
\documentclass[10pt]{article}

\usepackage[margin=2cm]{geometry}

\begin{document}



\section*{Introduction}

The conference is opened by N. Aveline who welcomes and thanks participants, honored of the presence of high representatives of SYSU and of the French government. She thanks the support of the EU and recalls that the Medium aid is not only academic, but also a tool to enhance dialogue, through stakeholder workshops for example. The feedbacks of experiments within three cities will be given during the roundtable. It also yield the signing of cooperation agreements between Chinese and European universities.


\subsection*{Xiao Haipeng, vice-president of SYSU}

Xiao Haipeng greets everyone and recalls the history of SYSU. The university has currently 4 campuses, and soon a 5th campus in Shenzhen. Very much efforts are done to exchange ideas with the world, and it is positioned as a world class university. For the urban issues in China such as traffic congestion and pollution, the study of sustainability is becoming increasingly important. In Pearl River Delta one tenth of National GDP is produced, but urban agglomeration also brings new challenges, that some were already encountered in Europe. He wishes great success to the partnership between SYSU, CNRS and other consortium members within this project.

\subsection*{Bertrand Furno, Consul General de France}


(In French) The consul welcomes everyone and is honored to open such a colloque. He thanks SYSU for the importance they give to the event, through the presence of the vice president. He recalls the urban issues China is facing, with the acceleration of urbanisation and the emergence of mega-cities, inducing the largest rural-urban migration in the world and strong unbalances. International experts can help therein, as with the 1000 talents programme. France and Europe are supposed to be able to bring expertise in green urbanism, eco-construction, which will be the next theme of the French-Chinese month. Exchanges between researchers are more important than ever to bring future solutions.


\subsection*{Laurent Bochereau, EU Representative in China}

European Union and China have a long history of diplomatic relationship, more than 40years, on peace, economy, and people to people relations. Science came a bit later, but was recently upgraded : a large cooperation on innovation now exists, investigating the conditions for innovation : how to promote it and what are the relevant funding mechanisms. It implies the implementation of Open Science and Open Policies, and more entrepreneurship. Urbanisation is a typically hot topic for innovation. A partnership in sustainable urbanisation was started 5 years ago, of which the precursor conference was in Foshan. Example of applications include nature-based solutions for restoration of urban systems and water cleaning. In conclusion, urbanisation is a key priority of EU-China partnership.


\subsection*{Li Yongjie, Chief Planner, Department of Housing and Urban-Rural Development, Guangdong Province}

The new phase of development of cities in China has brought sustainability as the biggest challenge. Due to rapid growth and expansion of cities, urban sprawl is not sustainable, causing an exhaustion of ressources and social problems. National government assigned high priority to subjects such as ecology in Urban Planning and participative planning. Results should be put into practice. One example is the 2010 masterplan that used the river network to conserve natural ressources and provide more space to residents for leisure and entertainment. In 2016, large achievements were reached in conservation, development of cultural activities, and the raising of public awareness. 



\subsection*{Xue Desheng, Dean of School of Geography and Planning, SYSU}

Pr. Xue gives greetings on behalf of SYSU School of Geography and Planning. The school has 4 departments, and is active in Planning in PRD, Guangdong, and other areas in China. In 1921 geography was introduced in SYSU. It has always fostered international collaborations. He wishes good luck for the conference.


\subsection*{Zhou Chunshan, coordinator of Medium in SYSU}

Pr. Zhou is pleased to have everyone here. He recalls the role of Zhuhai campus in supporting Medium activities. The Zhuhai planning bureau also brings great support.
he believes very robust research and progresses in Chinese are done here by European young researchers. He thanks the participants for joining the conference, and hopes there will be useful feedback on the various topics. The next year the conference will be hold in Datong, what should be the beginning of further collaborations. He wishes a good conference and an agreable stay in Guangzhou.



\section*{Keynote : Zhou Suhong, School of Geography and Planning, SYSU - Spatio-temporal behavior and the Smart city}

The smart city is reshaping daily life, through smart transportation, new ways of communication. The important features of Big Data are their Multi-scale, Real time, Individual nature. It can be illustrated with the example of GPS tracking data and resulting understanding of dynamics of the city. Real time route planning can be achieved. In Economics transformations are also important, with the raise of e-commerce, intelligent industries and sharing economy. For example a website of e-commerce recorded 120 Billion RMB in transactions on November 11th 2016.


This new era of transformation also brings new challenges. The example of an overcrowding of sharing bikes in a park in Shenzhen raises the question government readiness. It illustrates the changing of regulation mode : before, the government organized Urban Affaires through public companies. It is different from the delegation of service for sharing bikes, and the role of the public sector has to be rethought : the government is short of regulatory tools and short of real-time information on the system. New modes of cooperation have to be introduced, including public participation and self-regulation. It implies new Public Private Participation Modes. In the nineties, focus was on a Single Functional Service, in 2000 the service became more integrated, and in 2015 the emergence of the Smart City yields coordinated participation. Research can support coordinated participatory administration and personalized service in this new era, using frameworks such as Voluntary Geographical Information.


Important topics in Geography, Transportation and Land Use, GIS and Computer Science are related to the study of spatio-temporal trajectories, allowed by big data. Main corresponding research subjects include the study of Agglomeration and Clustering in Time and Space, i.e. the formation of Patterns ; the study of Correlation and Distance attenuation ; and the study of Processes and Dynamics. Examples of research can be given in Transportation and Mobility studies. The superposition of different mobility demand patterns yield the superposition of different processes, as shows a case study in Shenzhen with GPS records for Taxis, where reconstruction of Origin-destination matrix is possible. Behavior-based Urban research reconciles social geography and the study of spatio-temporal trajectories : we can for example observe clusters at different places in daily lives. Urban security and Public Health examples include models for Police-criminal interactions or patients-infectious disease interactions, for which spatio-temporal hotspot analysis and forecast models are crucial.

To summarize, the smart city is reshaping daily life, and Big Data is becoming ubiquitous. Corresponding new modes of public administration has to be elaborated, in link with various streams of new research that were shown.




\section*{Round Table : Feedback from Stakeholder Workshops}


This roundtable aims at giving a feedback on organized training workshops. The three main steps were training students, support them to organize, and collect output from the workshops. The context is participative planning, which aims at gathering different people having responsibilities in Urban Development.

The first point to raise concerns the recruitment process. From the experience in Hangzhou, Hangzhou Normal University was only recently implanted, and recruitment needs trust relationships. The second workshop was more successful as it relied on 6 months of fieldwork. Concerning the Zhuhai stakeholder workshop, a focused topic, the sustainability of the High-Tech zone, was important in choosing participants. The workshop was very useful for the Planning Institute, and allowed to grasp a wider definition of stakeholders. In Datong, the experience was quite interesting because not only students were implied, but also officials and practitioners, and it became halfway between a workshop and a training session. The role of social network (Guanxi) is determining in recruitment.


We can ask to what extent did the dialogue reached beyond the community, as different scales of participative dialogue were brought together. What did participants and the community actually gained from the workshop ? One beneficial point was to bring people from different backgrounds together, create mixed social groups, interacting in unconventional ways. Pedagogical skills were surely learnt, linked to how to relate to others in a very general way. One issue however was that it was not sure if someone would take the responsibility to go further after the workshop. People from different backgrounds had very different perspectives, and it was sometimes difficult to find common topics : in Zhuhai one participant kept fully locked on the achievement of the High-Tech zone. Sometimes some people gave wrong information. From the point of view of Zhuhai Planning Institute, it was a useful teaching of methods and approaches to facilitate public participation, quite similar to community planning model. The example of workshops in Xiamen illustrates the process of ``Building Together''. Place-making was in the center, and people became local community planners.


The openness of participants to solutions brought during the workshops was quite variable. In Datong there is a strong self-interrogation on how to get out of a resource-based economy, and a consequent uncertainty towards the future. They seemed quite open to new possible paths to follow. In Hangzhou, things were a bit different, actually the opposite, since the area was over-planned. Important stakeholders did not participate, and did not want to question a solution already designed.


The comparison of the Chinese context with other experiences can be done. The collaborative approach is dominant in western planning theory, but there is a large gap between theory and practice, and issues to make collaboration effective. Planning is much top-down in China, for historical or political reasons. Recently, public participation got more attention from the government. Participatory budgets exist in some cities. Technical devices may open new paths, through online mobilisation, but the design of the device will define and limite the scope of participation. Such workshops should bring sense for actors in cities. Generally,  many issues encountered in Europe are shared with China.


The implication of local communities is not straightforward, as for example in Zhuhai. It does not mean that the community has not a local role, as an initiative to coordinate between people in charge of the community with the government, and to enroll residents to join the community, was recently done, and the neighborhood committee had a significant role. There were also meetings with companies, to adapt planning to local needs. Some people implied in the district community thought they were ``not qualified enough'', but later representatives contacted back what lead to a more informal meeting.


Questions are now opened to the floor.


\paragraph{Question}

Different actors may have diverging interests. How did you manage potential conflicts ?

$\rightarrow$ They were some conflicts, and some could not be solved because of time. The aim is however not to solve conflicts, but taking note of it and find a way to move forward. Playing the role of moderators is however difficult for students. One must be careful also not to have ``too big'' players, i.e. large companies with considerable interests, as they bring much more bias in the process.


\paragraph{Comment}

There are in China conflicts of visions and preferences, but no conflicts of need as one can see in Europe. Is there a lack of competence of the High-Tech zone to attract educated people ? In any case local people don't want it, because it would yield raising prices. Everybody knows the problem but does not see it as conflict question.

$\rightarrow$ The observation of an issue, as here real estate, must be transformed into a strategy. It raises the issue of dialogue and what is the good level of abstractness to address such processes.


\paragraph{Question}

Are they shared interests among all cities, do they want all to go in the same direction ? 

$\rightarrow$ This question is quite hard to answer, as the paradigm of Medium-sized cities is that they want to become large. Datong for example is very specialized, but their paradigm of development is not yet redefined.


\paragraph{Question}

Were the workshop a place to discuss questions such as the role of urbanisation in modernisation and cultural issues ?

$\rightarrow$ The issue of Cultural Heritage was sometimes raised in Hangzhou, but not  a priority in workshops. In Datong planners think more in terms of mass tourism. It is a good example on how it is difficult to overcome sectorial differences.


\paragraph{Question}


What were the particular steps for the preparation stage : what documents to prepare, what specific topics to choose ? In Wuhan for example, a specific smartphone application was designed, including maps and 3D models. Also an important aspect is the willingness to pay of participants, was this issue raised ?

$\rightarrow$ Concerning the preparation of materials, some first must be prepared for organizers. Then an important workshop basic is that knowledge should be brought by participants. The objective is to invite informed people, with certain competences in complementary fields. Concerning financing, in Zhuhai, Xiangzhou district for example, district government gave funding for community planning and local communities used it for renovation.

\bigskip


To conclude, these experiments revealed more emerging questions than they gave answers, related to how to enhance dialogue, at which levels, for which plans and approaches, how to manage the relation with the government, and finally how to construct more robust policies.





\section*{Session : Spatio-temporal Behavior in Complex Systems}

\subsection*{D. Pumain, UMR G{\'e}ographie-cit{\'e}s - The SimpopLocal Model or what the period of emergence of cities can tell about urban dynamics}


The Evolutive Urban theory introduces rather simple models, but with the hope that they will informe concrete problems one day. It was the first time agent-based models were actually validated in a robust way. Archeological works reconstructing the population in Henan more than 2000 years ago shows that settlements already had a strong spatial structure and a hierarchy, typically a log-normal distribution of settlement sizes. To reproduce such stylized facts is the objective to multi-agents models of the Simpop series. The first one was introduced in 1996, and could reproduce basic stylized facts, but parameters values were mostly uncertain. Later many models were constructed within the Evolutive Urban Theory, of which the latest is the Marius model by Cottineau.

This presentation focuses on the SimpopLocal model, in which agents are stabilized settlements. The novelty compared to Simpop2 is that the creation os innovation is endogenous, and propagated between settlements. A logistic model of growth is used, the ressource being a function of innovation, and spatial interactions are translated with a gravity model. Some parameters are abstract cannot be extracted from data, and the use of intensive computation with the specifically designed software OpenMole is necessary. The principle of the OpenMole platform for this application is to define calibration objectives and to find optimal parameters to fit the data according to these objectives, in particular the distribution of settlement sizes. The use of genetic algorithms on a computation grid allowed a significant qualitative leap in results. The progressive hierarchization of a synthetic settlement illustrate the dynamics of the model. The variation domain for the estimated parameters was significantly reduced through calibration. A specific method, the Calibration Profile, allows to determine sufficient parameters to obtain the emergent patterns, and allows to suppress unnecessary parameters : in the SimpopLocal model, the lifetime of innovation is in fact not needed.

For the first time, necessary and sufficient conditions for such a simulation model to reach certain objectives were obtained, and the model was validated. Main pillars of the Evolutive Theory will be further explored with similar methods, namely (i) the hierarchical differentiation emerging from competition between cities ; (ii) the persistence of hierarchies, despite local fluctuations in sizes ; and (iii) the functional diversity in the geographical space emerging from innovation waves.

Three main stages can be found in the history of Urban dynamics : the agrarian economy, with a preponderance of local ressources, captured by the SimpopLocal model ; the market economy and the role of network returns (SimpopNet model) ; and a forthcoming phase which should imply Knowledge economy and Environmental intelligence. The co-evolution of cities can yield new modes of cooperation, although differences between cities are part of dynamics of the system.


\paragraph{Question}

What is the sample size used for the simulations ? $\rightarrow$ It stays a pure theoretical model, the parameters values obtained should be tested in a way by archeologists. Concerning sample size, the simulation can imply thousands to millions of models simulations.

\paragraph{Question}

Are the optimisation multi-objective, and were the plots Pareto fronts ? $\rightarrow$ yes





\subsection*{Zhan Qingming, Research Center for Digital City, School of Urban Design, Wuhan University - Urban wind path analysis and planning, supported by GIS and remote sensing}


With climate change a key issue for hot cities is that they that get warmer. Wuhan is one of such hotspots. As urbanization is accelerating, numerous issues for the environment are raising, from the use of air conditioners to health issues caused by pollution. Bringing nature in city centers should increase confortable zones without needing to use conditioners, and reduce pollution.


The framework of analysis relies on large scale data collection, with thousands of meteorological stations in one city, at an hourly temporal resolution. It is completed with open data such as Landsat data for land-use. Weather and Climate Forecasting models are used. Spatial analysis done are based on computation of the frontal area index. Wind speed is expressed as a function of height because of the roughness of buildings.

Case studies in Wuhan and Fuzhou are described. Showing the frequency of wind as a function of speed, direction, and temperature, we conclude that the distribution of directions depends on the season. The topography plays an important role : Fuzhou is surrounded by relatively high mountains. The temperature changes on long time scales, and the situation is getting worse in time. The definition of a Front Area Index unveils where wind can break through building lines. Wind path maps are obtained, and can be used to find potential air paths, as illustrated by a ventilation potential analysis for current and planned roads. The width of the new road must be larger if the transportation function is not the only one considered. Examples of suggested improvement in planning can be given, such as avoiding big blocks, or being careful on direction of buildings. In conclusion, the combination of heterogenous data and simulation tools allows a more scientific approach to planning, making the city greener and more confortable.


\paragraph{Question}

In dense areas, lot of buildings should been destroyed to implement the recommandation. How to incorporate wind paths in practice without demolishing ? $\rightarrow$ Demolition should be the last step. First plazas, green spaces, natural wind paths and corridors must be used. One example can be given in Hong Kong, where the wind can go through the building.

\paragraph{Question}

Investigating how the wind can relieve the pm particles and decrease pollution may be an important development of that research. $\rightarrow$ Indeed air pollution is the next target, more stations need to be installed.



\subsection*{M. Bida, C. Rozenblat and E. Swerts, Institute of Geography and
Sustainability, University of Lausanne - Modeling hierarchy of a system of cities as a result of the dynamics of firms’ interactions}

\paragraph{C. Rozenblat}

The basis assumption is the observation of hierarchy of city systems across the world at the macroscopic level, as it was mentioned before. The objective of the model is to reproduce these properties from the interaction of micro agents. At the micro level, transport choices, consumer choices and economies of scale are fundamental economic processes.


\paragraph{M. Bida}

The model includes economic agents, consumers and producers. Consumers are localize in settlements, and the producers locate themselves in the isotropic space. The dynamics translate a trade process, competition and profit accumulation. A producer has production cost and transport cost, plus fixed costs and prices, and aims to maximize its profit. The fixed costs imply economies of scale. Consumers adapt the quantity they buy according to price, given an allocated budget. To maximize their profits, producers will lower or raise prices depending on profit variations. If the profit becomes negative, they disappear. Economic assumptions on the market are a monopolistic competition, i.e. one type of good but different varieties ; no game theoretical behavior of consumers ; group of firms has no influence on global economy, so not on the allocated budget ; and entering and quitting the market is free. With static consumers settlements, the model can yield the emergence of hierarchy in producers spatial patterns. Experiments were done on the influence of varying parameters. For transportation cost, the higher the distance exponent is, the higher cumulated profit is. Economies of scale change the level of hierarchy. Demand distribution, i.e. how consumers spread across products, has an influence on hierarchy strength. Next steps in model development may imply to test how the inclusion of innovation influences model output.


\paragraph{E. Swerts}

To conclude, perspectives are given on a potential application of the model to Chinese cities. On the microscopic level, technological specificities, collaboration profiles, and local financial links must be empirically understood. Then at the mesoscopic level profiles has to been established for cities.



\paragraph{Question}

Is the value of the land in the production cost function ? $\rightarrow$ The value of land is not yet included in the model. Including the cost of labor would also imply taking into account more variables.





\subsection*{Shaojian Wang, School of Geography and Planning, SYSU - Current Situation and Countermeasures of Chinese low-carbon eco-metro area development}


In the context of climate change, China is a consequent contributor to carbon emissions, and low carbon metro-areas have been recently introduced to tackle this issue. There are some differences between low-carbon and ecology, in particular with an emphasis on functionality for the first and on bio-systems for the second. They have different economic implications, ecological conservation versus circular economy, and different spatial materialization, density versus spatial diversity. A compound concept would be a combination of both.

Development of low-carbon eco-cities has begun in the nineties with new districts creating an international experience. In 2000, these were taken as domestic reference. The low-carbon national reference should define the future trends of Chinese urbanization. The numerous requirements focus in particular on efficient spatial distributions. The Chinese government took in Copenhague in 2014 the engagement to reach the inflexion point in carbon emission in the next 5 years plan, but not all cities realized urgency of this issue. 57\% of overall measures are now implemented. Demonstration areas and Industrial parks that promote low carbon transformation have been built : 26 parks are now under construction, including Shouzhou and Tianjin.

However real low carbon cities are rather rare. Shenzhen is a model of such a city. Low carbon clusters and industry clusters must become the new goal of China's economy. Crucial issues include people taking profits on the concept, local government becoming indebted and too large cities. Agriculture and tertiary industries must integrate the potentialities of smart cities, smart industry, eco-cities and eco industries. A set of indicators to guide planners and developers must be constructed, and a particular care must be taken for the integration of concepts and theories into planning practices, including the definition of appropriate policies for the development of low carbon cities.

\paragraph{Question}

Low carbon seems a more top-down directive, what about the implication of local governments, do they implement policies given by the provincial government ?

$\rightarrow$ There are indeed several conflicts on this point. The central government has very ambitious targets, but how they can be break down into feasible targets at lower levels remains an open question.


\paragraph{Question}

Strategies developed include smart spaces devoted to industry and agriculture. Are there considerations on new models for buildings ?

$\rightarrow$ Low carbon policies have different guidelines in different domains. For buildings they advise low carbon materials, use of energy savings lamps, but No compulsory regulations are still lacking.





\subsection*{F. Pfaender, ComplexCity Lab, Shanghai University - Modelling with new urban micro data}

This presentation is about how to make cities in the lab from data, or how to construct macro models from micro data, and then apply them back at the microscopic scale. Understanding how microscopic interactions work can help to get a glimpse of the organisation of human traces. New type of data are not directly linked to traditional fields : the idea is that in one place, all kinds of data are simultaneously available.

Data that can be collected have particular properties, what can be summed up as ``Multi$^3$''. They are multi-scale, from the building and the neighborhood, to the city, and the system of cities. They are multilayer, implying different type of data and collaboration with different people. And they imply multiple timescales, with very different lengths. Harvest or capture data can be done in many ways, from the simplest to the hardest : datasets, use of API, scrapping, data sensing, small data. OpenStreetMap, with here the example of Guangzhou, is a directly usable dataset. The API answers to specific requests, here with the exemple of meetup.com. Webscrapping is a bit more difficult and implies getting data from the webpage. Constructing data sensors, such as handmade pollution detectors for mobikes, is powerful but costly and time-consuming. Finally, ``small data'' implies questionnaires and is the hardest to collect.

In summary, learning from data implies successive steps : (i) Data harvesting ; (ii) Data exploration ; (iii) Data modeling ; (iv) Data simulation. In echo Pandora's hope described by Bruno Latour, you need to be careful and conscious of what information you loose when filter between levels. A last example on Urban Settlements naming in China, mapped according to their last character, reveals spatial clusters. A crucial question is when do analysis show comparable situations, as for urban street patterns : is it possible to build a global framework of comparison ? The macroscopic models can then be used to go back to micro situations in which data are missing, and extrapolate data for cities that would not be included otherwise. It is one the the aims of this work that wants to be inclusive and integrative.


\paragraph{Question} 

How to choose the data to collect ? $\rightarrow$ It depends on the project. Every kind of data can be found at a location, sometimes all the possible data is collected.

\paragraph{Question}

The regional patterns of last character place naming, did it surprise someone ? $\rightarrow$ It is a very recent work. There is some spatial clustering, so there may be something that can be done.

\paragraph{Question}

How can we ensure accuracy of data ? $\rightarrow$ It generally comes from several sources and can be sometimes very noisy, there is no general way to ensure accuracy.



\subsection*{Yang Yue, Shenzhen University - Where to live in the arrival city: a spatiotemporal mobility approach}


This presentation focuses on where migrants live in the arrival city, with the case study of Shenzhen. A photo from the laboratory shows a construction site, next to the most expensive area in the city, that was two years ago still an urban village. Migrants could afford the place before, they had now to move to outskirts. The context of Shenzhen is particular with a rapid growth, it went from 6 to 10 districts in 9 years. The population is 20 million people, of which 77\% is from migration. The average age is 32,5.

The conventional approach to get data on migration phenomenon is through questionnaires. Here this approach uses transit smart card data. The socio-economic profile of people transiting is composed of younger, more graduate, middle-low incomes, and 85\% tenants. An example of rush hour peak flows visualisation shows the potentialities brought by the dataset. A statistical model is specified to estimate the origin-destination underlying matrix. Estimation results reveal where people live and work. Most of workplace are in inner zones of the city.

The residential map can be compared with the map of Urban Villages. The largest Urban village was located next to the most expensive area, and was recently reconstructed. The results of model estimation can show where do people relocate. The comparison of the relocation map with housing prices unveils that most go in an other urban village, and thus move from the center to outskirts. The clusters are highly statistically significant using Moran index for spatial auto-correlation. The distribution of increase in commuting time confirms that people must invest more time in commuting after the relocation.

The implications of this research are related to issues of urban vitality, urban gentrification and urban decline. Spatial segmentation is at the origin of raising social issues. Many debates on urban villages renovation are currently lead. City competition, between cities, but also competition between districts, is a driving factor of these issues. In conclusion, the urban renewal strategy and attitudes towards low-rent housing is key in sustainable development.


\paragraph{Question}

What are more moral or soft ways of urban renewal ? How to mitigate effects of such renewals ? $\rightarrow$ There are some debates among urban designers and architects on how renew Urban Villages, and there may be possible soft ways. The Government recognizes problem of high prices, but low rent housings are however located very far from the center.


\paragraph{Question}

How can people play a role in planning of the place where they live ?

$\rightarrow$ Many debates and experiments are done on the question, to encourage the promotion of participative urban planning.

\paragraph{Comment}

The example of village in Liede to be visited during the fieldwork raises the fact that the problem of the implemented scheme is that includes only land owners, and renters not included: they need to leave the zone.

\paragraph{Question}

This research may support government decisions, as they want people ``to go way'', and to leave ``slum places''. One must be careful on how it can be used. $\rightarrow$ Yes, indeed we need to be careful.






\subsection*{J. Raimbault, UMR G{\'e}ographie-cit{\'e}s - A macro-scale model of co-evolution for cities and transportation networks}


This presentation introduces an abstract modeling approach to the complex interplay between transportation networks and territories. One can consider the example of Pearl River Delta, were the currently built Hong-Kong-Zhuhai-Macao bridge will rebalance accessibility within the urban region, and ask how these new patterns will induce territorial transformations, but also how the development of networks is driven by territorial dynamics. Models of co-evolution have been rarely introduced, what could be due to compartmentalization of related disciplines. Different scales and ontologies are possible, and this work focuses on cities as agent whose interactions are network-mediated within a system of cities. The difference between submodels weak and strong coupling is recalled, and the co-evolution approach situated.


The rationale of the model is based on an extended Gibrat model, with interactions between cities as drivers of their population growth. Network growth is driven by flow demands, and the scale of network growth and the stochasticity level can be adjusted depending on the implementation. The model is tested on synthetic city systems, with a distance reinforcement specification. Hierarchy of populations and distances reinforces in time, but with inverse deviations from a scaling law. Accessibility patterns show a high level of trajectory crossing, what reflects strong changes in fate for medium-sized cities. Intermediate interaction ranges produce an optimal configuration for equity in accessibility distribution, but it corresponds to a maximal complexity of temporal trajectoires, meaning that it may be more complicated to implement in practice. First exploration of the physical network implementation reveal the capacity of the model to produce hierarchical networks.


Further developments will be directed towards the application on real city system, the French city system on two centuries with railroad data, and the Chinese city system in recent decades with the development of the High Speed Rail network. Comparisons should reveal the influence of governance context and planning level on coupled dynamics of networks and territories.


\paragraph{Question}

Is it reasonable to apply the model in such different contexts ? $\rightarrow$ It is the objective of such an approach, i.e. to construct models generic enough to be applicable to very different city systems, capturing both a certain level of universality and geographical particularities.



\subsection*{Round Table : Complexity and Knowledge of Complex Systems}

The round table aims to ask to each participant of the session his experience in production of knowledge on complex urban system, in particular the interplay between theories, models and data.


Models play very different roles in each approaches that was presented during the session. One important aspect is how they can percolate to policies, participate in a diagnosis and help to make planning scenarios, even they are not complicated enough to reproduce exactly real situations. On the other hand, theories are necessary to understand urban systems. New theories can nowadays be constructed through Big Data and new empirical knowledge.

On the issue of how to manage interdisciplinarity and achieve both quantitative and qualitative approaches, it is clear that the distinction between the two is artificial, and that the application of models and interaction with stakeholders helps to see these two faces of the same coin. Communication between disciplines remains however a big challenge. From the point of view of data science, a new role of mediator may be emerging.





\end{document}











