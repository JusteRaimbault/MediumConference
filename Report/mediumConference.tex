


\section*{Introduction}

Thanks word by N. Aveline in Chinese ; switches to English.

Honored of high representatives of Sysu and French governement. Support of EU.

Medium aim ; not only academic (business, policy makers). Tool to enhance dialogue : stakeholder workshop. Dialogue necessary between different components of population within cities. Experiments within three cities : feedback during the roundtabe. Signing of cooperation agreements (CaFoscari, Lausanne, Paris 1).


\subsection*{Haipeng Xiao, vice-president of SYSU}

Greetings ; history of SYSU. Currently 4 campuses, soon 5th campus in Shenzhen. Very much efforts to exchange ideas with the world ; world class university.

Urban issues in China : traffic, pollution etc. Sustainibility increaglingly important. GDP of PRD 1/10 of National GDP, but many problems and challenges. Same problems in Europe : thanks experts that bring knowledge.

Partnership SYSU, CNRS, etc. mutual understanding, collaborate effectively ; wishes great success

\subsection*{Bertrand Furno, Consul General de France}

(in French). Welcome adresses. Honor to open such a colloque.  Thanks Sysu and the importance they give (presence of vice president). Explosion of urbanisation, emergence of mega-cities. Country in mutation, largest rural-urban migration ; unbalances. Negative externalities. China uses international experts : 1000 talents programme. France and Europe bring expertise in green urbanism, eco-construction. Next theme of French-chinese month. Now exchanges between researchers, will bring future solutions. Co-financement by Chinese governement of Euro-Chinese projects. (no more direct financements by Horizon2020). Wishes great success, thanks everyone.


\subsection*{Laurent Bochereau, EU Representative in China}

Thanks everyone. EU-China long history of diplomatic relathionship (40years) : peace, economy, people to people. Science a bit later, recently upgraded : innovation cooperation (last week third meeting). Conditions for innovation : how to promote, funding mechanisms. Open Science, Open Policies ; more entreupreunership. Urbanisation hot topic. Flagship collaboration. More mobility to China : very low number of Europeans young researchers. Clean energy, food and agriculture. Intensify relationships between Universities.

Urbanisation : key point. Sustainable urbanisation partnership (5 years ago). Initiatives, cities on both sides. Conference in Foshan, precursor. Platform : Urban EU-China ; at different levels, from States to cities. Exchange of best practices. Last call for Horizon2020 ; many topics in urbanization, with participation of China. Ex. Nature-based solutions for restoration of urban systems ; water cleaning. Conclusion : urbanisation key priority. Seems very good research.

\subsection*{Yongjie Li, Chief Planner, Department of Housing and Urban-Rural ?}

Welcomes. New phase of dvlpmt of cities in CHina, biggest challenge is sustainibility. rapid growth, expansion. sprawl not sustainable, running out of ressources, social problems. Conference focused on industries. Production and ecology in Urban Planning. Participative planning. National governement assigned high proiority. results should be put into practice. Original masterplan. 2010 : learn from EU/Meiguo, river network to conserve natural ressources, provide more space to residents for leisure and entertainement. 2016 : lead in conservation. Cultural activities, raise public awareness. 

Today important international conf. in Sysu. Opportunities by one belt one road initiative. 



\subsection*{Desheng Xue, Dean of School of Geography}

Greetings on behalf of DiliKexuexueyuan. (courageous with the weather). Presents School of Geography and Planning. 4 dpts. Planning in PRD, Guangdong, and other areas in China. 1921 fundation of geo in Sysu, always international collaborations. Many projects, waits forward. Wishes good luck.


\subsection{Chunshan Zhou, coordinator of Medium in Sysu}

Greetings. Pleasure to have everyone here. Zhuhai, Zhuhai campus. ZHuhai planning bureau great support. Very robust research, progresses in Chinese. Thanks participants, joigning the conf ; feedback on topics. Next year Datong, new beginning of further collaborations. Wishes good conf. and good stay in Guangzhou.



\section*{Keynote : Zhou Suhong, Spatio-temporal behavior and the Smart city}

Smart city reshaping daily life. New smart transportation. New way of communication. Big Data : Multi-scale, Real time, Individual, Tracking Data. Example : GPS tracking data and dynamic of the city. Real time route planning. Economic : e-commerce, intelligent industry, sharing economy. ex TMall 120Billion rMB on Nov 11th 2016. New era of transformation, new challenges. Example of crowded of sharing bikes in Shenzhen: was the government ready ? Example of changing of regulation mode : Government -> Urban Affaire -> Company -> Public ≠ Delegation of service for sharing bikes, plus role of public : giv. short of regulatory tools, short of information. New mode of cooperation, public participation and Self-regulation. Higher level for the company. New PPP (Public Private Participation Mode). 1990 : Single Functional Service -> 2000 Some integrated service -> 2015 Smart city, coordinated participation. How can research support coordinated participatory administration, personalized service in the new era. Framework : VGI... . Main topics in Geography, TranportationLandUse, GIS COmputer Science : in time, increase in trajectories, big data. Main research fields : Agglomeration and Clustering in Time and Space :Patterns ; correlation and distance attenuation : Spatio-temporal Correlations (Near repeat, Distance decay, colocation) ; Processes and Dynamics.

Examples of research : Transportation / mobility. Superposition of different needs of mobility : superposition of different processes. Case study in Shenzhen : Taxi's GPS records : O/D ; fit gradient decay

Behavior-based Urban research : social geography + spatio-temporal trajectories. : cluster at different places in daily lives ; segregation (// travail Julie) Zhou 2015.

Real-time Modeling.

Urban security and Public Health. : Polices-criminal interactions ; patients-infectious disease : interactions in space and time. $\rightarrow$ Spatio-temporal hotspot analysis and forecast models. MAS for criminology behavior (example Netlogo).

Summary : (i) SMart city reshaping daily life ; (ii) Big Data ; (iii) New modes of public administration ; (iv) New research.

(Natacha) thanks for sharing theoretical and empirical results.





\section*{Round Table : Feedback from Stakeholder Workshops}


(Sabine) Welcomes. Sketch what achieved ; how can capitalize on that. Governance, other stakeholders. Introduces participants. Program.

(Erik) Organized training workshops, how to bring people into a dialogue. Three main steps : training students, supported them to organize, collect output from workshops. Participative planning : people responsible of urban dvlpmt see situation the same way, so they can speak together. Pilot project : what type of method work in the cultural context of China.

Who does bring into the process : recruitment process.

(Valentina) Experience in Hangzhou. HZNU only recently implanted. To recruit, trust relationships. First workshop : not very local stakeholders ; Second : rely on 6month of fieldwork. Incremental recruitment. 

(Chenyi) For the Zhuhai stakeholder workshop : how to choose, how to convince. Topic : sustainibility of Hi-Tech zone. Organisation where know people ; others meet directly. How to convince : topics, brainstorm. Other had to situate the context. Difficult to recruit people from the community. 

(Zhuhai Planning Instiute) Importance of workshop. Forums in which they are interested in. Official wechat account. Wider def of stakeholders. Structuring of the zone ; suggestions.

(Judith) In Datong. Very interesting because not only students (no research in Datong). Halfway workshop/training. Prof. Chen very good Guanxi. Officials : members of planning Bureau, responsible for financing. University located near to new gov. building. 

(Erik) Dialogue beyond the community : different scales of participative dialogue brought together. What did they gain from the workshop ?

(Judith) difficult because Datong just happened. People could stay all the time : other dimension than content (new notions : participative, sustainibility) bu method : bringing people together from different background, unconventionnal, brainstorm, mixed social groups. Concrete organisation impact. (Erik) Pedagogical skills learnt : carry out a meeting in an other way. How to relate to other in a very general way.

(Valentina) People not trained before. Diverse results. Notion of ressource. Even if people interested in the approach, who takes the responsibility in the end, to go further.

(Erik) New type of questioning : do we have governance needed ?

(Valentina) Second workshop, follow-up. Swot : built a little bit more, settle wechat group, but nobody takes the responsibility : who has to lead ? in the end, nobody goes on with what emerged. 

(Erik) Not mandated, only showing where could go. Zhuhai planning institute : Chenyi, dialogue ?

(Chenyi) People from different backgrounds, different perspectives. Difficult to find common topics : guy speaking only of the achievment of HighTech zone. Sone people gave wrong information.

(Planing Institute) Very honored to be part of workshop. Teaches methods and approaches to facilitate public participation. Similar to community planning model ; different from traditional planning. Lack of parking space / public space / public road network. Questions and issues brought by residents as third party : use professional knowledge to look into this issue. Issues more urgent. Projects under way, already results. Draw experience in the future : replicable model ?


(.) WOrkshops in Xiamen. ``Build Together workshop''. Most important places.  Placemking to begin workshop. Open place provided by governement. People become local community planners. What can the community be ? What they care about has been solved. Between governement and community. What can planners bring to community to be better.

(Sabine) Interest in OpenNess. Judith ?

(Judith) Adapted a lot methods for Datong. Context : very important geo, socio-economic situations. Datong : self-interrogation, no more ressource-based economy. Uncertainty towards the future, no plan for the future. Diverse paths possible to follow. 

(Valentina) In Hangzhou, a bit different. Counter-example ? New area, not implemented, should be open ? Actually the opposite, issue was overplanning. People important did not participate. Even feedback. Don't want to question solution already designed ?

(Weilang Zhang) People affected ? difficult to involve community ?

(Sabine) Perceptions of China ; comparison with experience Nicolas ?

(N Douay) Collaborative approach dominant in western planning theory ; but gap between theory and practice. Issues to make collaboratoin effective. Very top-down in China (history or political). recently, for social accessiblity, public participation got more attention from gvt. Ex exhibition : collaborative centers. Participatory budgets in some cities. But limited to basic info. Technical device ? (online mobilisation). technical question at the beginning ; design of device will define and limite the scope of participation. Ex. Changdu//Paris (participatory budget). Such workshop develops meaning for actors of cities. Well organized, // literature : who ? Deliberation ? Why ? (impact of the decision) Many issues shared with China.

(Erik) why did not see communities in Zhuhai ? 

(Planning Institute) Coordinating resident to join, coordinate with person in charge of community. What to do ? coordinate with gov. What to do and how. leverage neigh community. Now wechat account. Neigh commitee brought a lot. Did answer good ?

(Erik) Complex cities need something higher. bringing all responsible at a higher level.

(Planning Institute) Meeting with companies : adapts to local needs (implantation of utilities).

(Valentina) Add something to this : coincide with a district. [not true!] Other activities than the companies. People thought ``not qualified enough''. Representative contacted back : cf informal meeting ; people preserving the area. Just connect the dots.

(Erik) Opens to the floor


(Q : Celine Rozenblatt) Very different interests : potential conflicts, how did with them ?

(Chenyi) They were conflicts : man saying Zhuhai facing aging problems, disagreed why support aging industry ; did not come to common point because of time.

(Erik) Not solving conflicts, but taking notes of it, find a way to move forward.

(Judith) Role of moderators : difficult for students. 

(Valentina) Big players too big : one redirects totally, puts interests of his company.

(Q : Yinghao) comment : conflict of visions : preferences, but no conflicts of need, as one can see in Europe. Lack of competence of hightech zone to attract educated people ? But don't want, because would be raising prices. Everybody knows the problem but does not see it as conflict question.

(Erik) Observation of an issue (ex real estate), how to tranform into a strategy. Issue of dialogue. Level of abstractness to address these processes ?


(Q ?) Sharing interest among all cities : same direction ? 

(Judith) Hard to answer. Paradigm of Medium-sized cities : want to become large. Datong very specialized, but paradigm of dvlpmt not redifined.

(Erik) Pb of ressources. Intuitions emerging in short dialogues. 

(Q Daniele Venice) Urbanisation is key in modernisation - workshop to discuss this kind of question ? 

(Valentina) Cultural Heritage issue. Not really emerged as a reflexion there. Issue for people : present in their agenda, but not in workshop.

(Erik) and in Datong with tourism ?

(Datong) Devaluate daily life, but think in mass tourism (although practice alternative activities - hiking).

(Sabine) example on how difficult to overcome sectorial differences.

(Q : ? Wuhan University) Preparation stage : what documents prepare, people speak of the same thing. In Whuhan, application. maps, 3D models etc.

(Q : ?) Participatory planning important in China. Purpose not only to know their need, but also their willingness to pay.

(Valentine) Preparation of materials : first prepare for organizers. Follow-up in Hangzhou. first workshop basic : knowledge brought by people. Second : maps, Swot

(Erik) Objective : invite more informed people, some with certain competences in fields. 2 Q : financing.

(Valentina) Ex second Hangzhou. Involving companies. 

(Erik) Focus not on wishlist, but three pillars of sustainibility. how economy balances social and environmental needs. Last comments ?

(Planning Zhuhai) Who pays ? how put in action ? what they care about most. Already survey in Xiangzhou : district gov. already gives money for community planning. Local Commities uses fundings to renovate. Activities ? Fund called beautiful three corners, to renovate community. For participatory planning, provide local people and leverage funds provided by gov.

(Erik) Shred experience important. More emerging questions than answers : how to move on dialogue : levels, plans and approaches, relation with govt ; construct more robust, better policies. Some authorities more receptive than others.
Thanks the panelists, the audience, continue discussion later these days and the coming.


(Natacha) Thanks for chairing - tomorrow 9.





\section*{Spatio-temporal Behavior in Complex Systems}

\subsection*{D. Pumain}

Simple models, hope will informe concrete problems one day. First time validate agent-based models.

Example of Henan (Liu 96) : already spatial structure of settlements : competition. Strong hierarchy of settlements sites (log-normal distribution).

The aim of simulation multi-agent models is to reproduce the dynamics of settlements. Done thanks to OpenMole.

Part of the series of Simpop Models. First one : Bura et al., 96. STylized facts, but not sure of parameters values. Successive generation within Urban Evolutionary Theory. Latest : Marius.

The SimpopLocal : agents are stabilized settlements. Novelty / Simpop2 : innovation creation is endogenous. propagated between settlements. Logistic model of growth, resource function of innovation, gravity model for interactions. Some parameters cannot be infered from data.

Constraints : stylized facts : size distribution of settlements, maximum size $10^3$, time to reach 4000 years.

Principle of the OpenMole platform : define calibration objectives, find optimal parameters. Use genetic algorithms on grid $\rightarrow$ huge qualitative leap.

Illustration progressive hierarchisation. Estimated parameters : variation domain significantly reduced (even on Pareto front). Specific method : Calibration Profile. (plots). suppress unuseful parameters (here life of innovation is not needed).

For the first time, necessary and sufficient conditions ; model validated. Hope to validate main points of Evolutive Theory with similar methods :
\begin{itemize}
\item Hierarchical differentiation emerging from competition
\item Persistence of hierarchies, despite local fluctuation
\item Functional geodiversity emerging from innovation waves.
\item ?
\end{itemize}

Three stages in Urban dynamics.
\begin{itemize}
\item Agrarian economy, local ressources (Simpoplocal)
\item Market economy, network returns (Simpopnet)
\item Matters of network (global problems) : use it : Knowledge economy, Environmental intelligence : SimpopClim
\end{itemize}

Urban co-evolution : new modes of cooperation.
Although differences between cities are part if dynamics of the system.
Ethic always necessary.
(ICT : skipped).

cf site geodivercity, and book !

\paragraph{Question}

Sample used for the ABMS ?

$\rightarrow$ Arbitrary. Stays pure theoretical model. Hope found parameters could be tested by archeologists. other kind of validation.

But number of repetitions ?

$\rightarrow$ Few hundred. Clem : thousands. Base Elfie : trajectories of Chinese cities - SimpopSino ?

\paragraph{Question Polli}

Objective : multi-objective, Pareto front. are the last presented ?

$\rightarrow$ yes

(Zhou thanks)


\subsection*{Zhan Qingming : Urban Wind path analysis}


Hot issue for hot cities, that get warmer !

Wuhan one of hotspots. Why some cities hotter ?

\paragraph{Background}

Urbanization accelerating process. not environmental friendly (air conditioners), up to health issues. Cities as ``pancake making'' : superposing layers.

Graph : Wuhan temperature, with confortable region. if bring nature in city centers, increase confirtable zone without use of conditioners, and reduce pollution. Incoming issue as size will go bigger.

\paragraph{Framework}

Phenomenon $\rightarrow$ Analysis $\rightarrow$ Planning $\rightarrow$ Implementation.

Thousands of stations in one city ; hourly : big data.

Open data : Landsat, etc.

Use of Weather and Forecasting model. Spatial analysis : frontal area index.

Wind speed as function of height because of buildings. roughness.

\paragraph{Case study}

Wuhan, and Fuzhou (hotest city).

plot : frequency as function of speed and direction ; and temperature. Distribution of directions depends of season.

Topography. Land SUrface Temerature. Fuzhou surronded by 100m high mountains.

Teperature changes in time, situation getting worse.

Winf roses at each station.

Front Area Index : where wind can break through.

Results : Wind path mapping. Different types of wind. See breeze. River as windpath. : find potential air paths. (first, second level).

Ventilation potential analysis for current and planned road. Width of road larger if not only transportation function. plus orientation. problem orientates solution.

Sample examples of suggested improvement : Avoid big blocks, put in direcion.

Planning implementation : small steps. Control, give planning guidelines.

\paragraph{Conclusion}

Heterogenous data and simulation tools : more scientific approach to planning to make the city greener and more confortable.

Limitations : link between levels.

Planning and Design.

\paragraph{Question (ZH)}

Large area, lot of buildings destroyed, done only in theory, how to incorporate in practice without demolishing ?

$\rightarrow$ demolition is the last step. Three step process. Should be demolished ? One example in Hong Kong : wind can go through building. Use plazas, green spaces. Natural wind paths, corridors.

Thought of it in planning, along the coast, good ventilation.

$\rightarrow$ coastal cities have more ressources compared to inner cities.


\paragraph{Question}

Major quality major concern, how the wind can relieve the pm particle, look forward for extensions.

$\rightarrow$ air pollution next target. More stations, Wuhan is big.

\paragraph{Question (ZH)}

Macro-design. Have data on wind penetration rate. Zhuhai, different from Hong Kong. Do have such data ?

$\rightarrow$ some data on all, some on some cities only. Upper bound on density : 50\%. Some developers build huge properties along the coast : impact on ventilation. Consider danger on building long building along the coast.



\subsection*{M. Bida, C. Rozenblatt and E. Swerts}

(Celine) New work started three months ago. In frame of Medium and FNS LOGIICCS. setup and objectives

Start form observation of hierarchy of city systems across the world. (macro observation) (cf presentation Denise)

objective: model of micro agents, able to reproduce macro facts.

remainder micro, meso, macro. : micro levels, with transport choices, consumer choices and economies of scale. Objective : hierarchy.

(Mehdi) Theoretical model. ABM. Economic agents : consumers and producers. consumers in settlements, producers random. isotropic space.

Dynamics : trade process, competition and profit accumulation. Flowchart : Producer has production cost, transport cost. plus fixed costs and prices : profit.

Fixed costs -> economies of scale. Cnosumers adapt the quantity they buy according to price, given an aoolcated budget. If two producers : B has higher price. Consumers buy from all producers : market shares.

(Q : local equilibrium in time ?)
(Q : no distance effect in buying ?)

Producers : maximize profits. lower or raise prices depending on profit increase. When profit negative, dies.

Hypothesis of the market : 
- Chamberlin's monopolistic competition. One type of good but different varieties. (Rq : // circ Eco !)
- no game game theoretic behavior
- group of firms has no influence on global economy (so not on allocated budget)
- entry/quit market : free.

Static consumers settlements. Emergence of hierarchy.

(Q : no relocation dynamics - would be very different ?)


\paragraph{Results} 

Parameters variation.

Transport cost : $d^{\alpha}$ concave, linear and convex. The higher distance exp, the higher cumulated profit.

Influence of ecnoomies of scale : changes hierarchy level.

Demand distribution : how consumers spread across products (cheapest, or distribute among all prices).
influence on hierarchy strength. 

(Rq : les plusieurs régimes sont intéressantes)

\paragraph{Discussion micro-macro}

transport costs protects close producers.
Econmies of scale favorize already settled.
Early producers have advantage.


Combination of params ? maximize hierarchy.


\paragraph{Next steps}

How innovation improves.

(Q : population constant in each settlement ?)

(Elfie)
Application to Chinese cities ?

Micro level : Techno, collaboration, finnancial links ; Then meso level.

\texttt{CitaDyne : http://www3.unil.ch/wpmu/citadyne-news}


\paragraph{Question Natacha}

Delocating production ?

Value of the land in production cost ?

$\rightarrow$ not yet for value of land. Delocation : rather new. (Celine) lower cost of labor : more variables.




\subsection*{Shaojian Wang}


Context : climate change. China big contributors to emissions : recent focus from different perspectives on how to manage different perspectives.

Low carbon metro-area.

%\paragraph{Context}

Difference between low-carbon and ecology. emphasis on functionality vs biosystem. different economic implications : ecology circular economy.
Sptail difference : density vs spatial diversity.

Compound concept : combination of both.

Development of low-carbon eco-cities. in 90s, new districs : international experience. 2000s :domestic reference.

low-carbon new national ref. Future trend of Chinese urbanization.

Requirements (many).

Focus on efficent spatial usage.

Promesses of ZH gov (2014) Khopenhague. In 5 year plan ; inflex emission.

but not all cities realized urgency of this issue. Current issues and achievments.

57\% of total impplement. SOme publish first batch on pilots. 8 cities. 

Demonstration areas. Industrial parks. promote low carbon transformation

26 parks under construction ; includes shouzhou, tianjin that passed first time assessment.


Lot of parks, but real low carbon cities are not that much. Shenzhen raw model.

low carbin clusters and indutrsty clusters. new goalpoint of CHinas economy.

On pilot projects : three categories.

Problems :
 - people taking profits on the concept.
- governements become endetted.
- still developing industries on low end, not green industries.
- large cities
(rq : morphology / consumption : pb : les gens utilisent single dimensionnal ; tout la q est de cpaturer le max de dimensions significative à la q...)

Make sure enough space for construction.


Agriculture and tertiary industries : integration of smart cities/smart industry / eco cities eco industries.

central gov. propose smart spaces : prodiuction, living, ?. integarted into integration model. facilitate colboration baks, gov, companies. attract more private capitals : shift from real stat dev into smart low carbon industry.


Set of indexes : how to guide planners/developers. Integarte concepts and theories into planning practices : tax/land/dmeolition policies, need to be formulated properly.

Last but not least, policies for dev low carbon cities.

\paragraph{Q Liao Liao}

did not pay attention to policies. Low carbon is more top-down. What about local gvernements : cities, township, villages.
do they implement policies from provincial gov ?

$\rightarrow$ yes difficult. many conflicts. Central gov. has very amitious targets, but how can be break down into feasible targets at lower levels remains open question. Many areas received standards ; do have policies but just guiding. for local gov, still pay more attention to economic (GDP) for small and medium cities. Policies are better implemented in larger cities. It takes time.

\paragraph{Q Polli}

Strategies : smart spaces devoted to industry and agricultures. Consideration on how model new buldings ?

$\rightarrow$ For low carbon policies, different guidelines on different domains. bulding : low carbon materials, use of energy savings lamps etc. No compulsary regulation


(Zhou) For sustainibility, both importance of research and policy making : more discussions linked to more modeling. Suggestions for further discussions : what about connectoin with physical aspects. Impact of physical georgaphy context ? Connect physical context (presentation wind path) with toy models, would give more reasonable models. More taking planning in instituionnal context (very important for China).
(rq : misunderstainding on aims of ABMS : dont want real world simulation)
for wind path : more connection with human geography ?
lot of info.


[Tea break - back 11h10]




\subsection*{F. Pfaender}

Disclaimer : not a geographer, data scientist working on Urban data.

How make cities in the lab from data : macro model from micro data ; then application on micro.

Micro : understand the micro level ? how micro interactions work, what can get a glimpse of human traces organisation.
Tyoe of data ; not linked to tradiotionnal fields : in one place, all kinds of data. example : air conditioning car -> heat -> people working move -> change economy.

Multi$^3$ : scale - from building, neighborhood, city, system of cities.
Layer : different type of data ; collaboration with different people.
timescales : very different lengths.

In China, realize that very dififcult to get data. system changes, way of collecting data. collboration different people. 

Harvest/capture data.
from the simpliest to hardest : Datasets, API, scrapping, data sensing, small data.

Guangzhu OSM.

API. exemple meetup.com.

Webscrapping : example Dajong Dianping - like yelp. : point of interest. get data from the webpage. Use it in papers ? grey area.

Data sensor : example box pollution. Not good, but comparison ok. cheap electronics very useful. but takes a lot of time. ex equipment mobike.

Small data : questionnaires.

(storing the data : skipped).

Learning from data : 
 - Data harvesting
 - Data exploration
 - Data modeling
 - Data simulation.
 
macro level : with everybody.

Latour, Pandora's hope : know what you loose when filter

Urban setllements : last character : spatial clusters.

Comparable situations ?

Urban street patterns : global framework of comparison ?

Then go back to micro situations where you miss the data. : have data for cities which dont have, dont forget cities. small settlements. (rq : different regimes).
aim of this work that embeds everybody.


\paragraph{Q (China institute fo clean energy} 

How choose the data to collect ?

$\rightarrow$ traces. on what some work ; every data that can be found at a location. ex. mobike go see them. Either a specific project, or everything.

\paragraph{Q D. Pumain}

regional patterns of last character.
did surprise someone ?

$\rightarrow$ recent work. geographical character, so clustering normal. searchiong for collabs with demographics, literature, etc. There is some clustering, so there may be something that can be done.

\paragraph{Q}

Accuracy of data ?

$\rightarrow$ several sources ; very noisy.



\subsection*{Y Yue}


WHere to live in the arrival city : migrants.

case study on Shenzhen. Photo from lab, campus in green, construction site ; next to most eexpensive area. two years ago, urban village !

Even for researcher too much expensive. : settle far, needs to drive to the campus. Doubled in ? years.

Migrants could afford before, now where do they have to go.

Context of Shenzhen. From 6 to 10 district in 9 years.

Administrative context : SHenzhen urban districts, restricted zones.

facts : 20 million population. 77\% migration ; 32.5 average age. Half of salaries in rent.

Conventionnal approach to get data : questionnaires.

Approach : trsanit smart card data. 

Profile of transiters : younger, graduate, middle-low income, 85\% tenants.

Rush Hour peaks.

bus smart card data : no get down data.

(model specification ?)

results : where people live and work. work most in inner zones.

Compare with map of Urban villages (density).

Largest Urban village next to most expensive area. (photos.. : commodities ?)

-> Reconstructed. Where do people reolcate ? : relocation heatmap, compared to housing prices : most go in an other urban village.

Relocations : move from the center to outskirts.

Clusters : local and moran I : clusters highly significant.

Increase of commuting time (PDF)

Implications : Urban vitality, Urban gentrification, Urban decline (photo : central area very expensive but empty), spatial segmentation, social issues.

Research programs, drama, articles : debates on Urban village renovation.

Impcation : city competition. (? outperforms HK) ; competition between districts. Migrants are driving force. Need to rebalance, what about central areas.

COnclusion : urban renewal strategy and attitudes towards low-rent housing is key to sustainable development.

\paragraph{Q}

More moral/soft way of renewal ? how to mitigate effects ?

$\rightarrow$ debates urban designers/architects on how renew urban village. possible soft ways. Governement recognizes problem of prices, but low rent housings are very far from the center.

\paragraph{Q (italien)}

how do people play a role in planning of the place where they live ? bottom-up

$\rightarrow$ some debates and experiments - promotion of participative urban planning. (better answered by an urban planner expert).

\paragraph{Comment Natacha}

Liede village in Gangzhou to be visited : problem of scheme is that includes only land owners ; renters not included, need to leave the zone.

\paragraph{Q}

support government decisions : they want people ``to go'' : want people leaving ``ugly places'' : beware, this research can give support to govt

$\rightarrow$ yes, must be careful.































